% Notes and exercises from Linear Algebra and Geometry by Dieudonne
% By John Peloquin
\documentclass[letterpaper,12pt]{article}
\usepackage{amsmath,amssymb,amsthm,enumitem,fourier}

\newcommand{\sect}{\cap}

% Theorems
\theoremstyle{definition}
\newtheorem*{exer}{Exercise}

\theoremstyle{remark}
\newtheorem*{rmk}{Remark}

% Meta
\title{Notes and exercises from\\\textit{Linear Algebra and Geometry}}
\author{John Peloquin}
\date{}

\begin{document}
\maketitle

\section*{Introduction}
This document contains notes and exercises from~\cite{dieudonne}.

\section*{Chapter~III}
\subsection*{Section~1}
\begin{exer}[4]
Let \(V,W\) be a pair of supplementary subspaces of~\(E\). Every subspace~\(U\) containing~\(V\) is the direct sum of~\(V\) with~\(U\sect W\).
\end{exer}
\begin{proof}
If \(u\in U\), then \(u=v+w\) for some \(v\in V\) and \(w\in W\), and \(w=u-v\in U\). So \(U=V+(U\sect W)\), and \(V\sect U\sect W=\{0\}\).
\end{proof}

% References
\begin{thebibliography}{0}
\bibitem{dieudonne} Dieudonn\'e, J. \textit{Linear Algebra and Geometry.} Hermann, 1969.
\end{thebibliography}
\end{document}
