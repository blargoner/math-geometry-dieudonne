% Notes and exercises from Linear Algebra and Geometry by Dieudonne
% By John Peloquin
\documentclass[letterpaper,12pt]{article}
\usepackage{amsmath,amssymb,amsthm,enumitem,fourier}

\newcommand{\R}{\mathbf{R}}

\newcommand{\iso}{\cong}

\newcommand{\sect}{\cap}
\newcommand{\after}{\circ}

\DeclareMathOperator{\End}{End}
\DeclareMathOperator{\GL}{\mathbf{GL}}
\DeclareMathOperator{\GA}{\mathbf{GA}}

\newcommand{\inv}[1]{#1^{-1}}
\newcommand{\kerz}[1]{\inv{#1}(0)}

% Theorems
\theoremstyle{definition}
\newtheorem*{exer}{Exercise}

\theoremstyle{remark}
\newtheorem*{rmk}{Remark}

% Meta
\title{Notes and exercises from\\\textit{Linear Algebra and Geometry}}
\author{John Peloquin}
\date{}

\begin{document}
\maketitle

\section*{Introduction}
This document contains notes and exercises from~\cite{dieudonne}.

\section*{Chapter~III}
\subsection*{Section~1}
\begin{exer}[4]
Let \(V,W\) be a pair of supplementary subspaces of~\(E\). Every subspace~\(U\) containing~\(V\) is the direct sum of~\(V\) with~\(U\sect W\).
\end{exer}
\begin{proof}
If \(u\in U\), then \(u=v+w\) for some \(v\in V\) and \(w\in W\), and \(w=u-v\in U\). So \(U=V+(U\sect W)\), and \(V\sect U\sect W=\{0\}\).
\end{proof}

\subsection*{Section~2}
\begin{exer}[1]
If \(p,q\)~are the projections corresponding to a direct sum \(E=V+W\), then \(p,q\in\End(E)\) are such that \(p^2=p\), \(q^2=q\), and \(p+q=1\). Conversely, if \(p\in\End(E)\) is such that \(p^2=p\), then \(E=p(E)+\kerz{p}\) is a direct sum. Moreover, if \(q=1-p\), then \(q^2=q\), \(q(E)=\kerz{p}\), and \(\kerz{q}=p(E)\).
\end{exer}
\begin{proof}
For the forward direction, we know \(p,q\in\End(E)\) and \(p+q=1\) (3.2.2). It follows that \(p^2=p\after(1-q)=p-pq=p\) and \(q^2=(1-p)^2=1-p=q\).

For the converse, \(p+q=1\), so \(E=p(E)+q(E)\). Also \(pq=p-p^2=0\), so \(p(E)\sect q(E)=\{0\}\) and \(q(E)\subseteq\kerz{p}\). If \(x\in\kerz{p}\), then \(q(x)=x\), so \(x\in q(E)\). Hence \(q(E)=\kerz{p}\) and similarly \(\kerz{q}=p(E)\). Finally \(q^2=q\) as above.
\end{proof}

\begin{exer}[2]
If \(W\) and~\(W'\) are both supplementary to~\(V\) in~\(E\), then \(W\) and~\(W'\) are isomorphic.
\end{exer}
\begin{proof}
If \(p\)~is the projection of~\(E\) onto~\(W'\), then the restriction of~\(p\) to~\(W\) is an isomorphism from~\(W\) to~\(W'\).
\end{proof}

\begin{exer}[3]
If \(E=V+W\) is a direct sum with inclusions \(i:V\to E\) and \(j:W\to E\), and \(v:V\to F\) and \(w:W\to F\) are linear maps, then there is a unique linear map \(u:E\to F\) with \(u\after i=v\) and \(u\after j=w\).
\end{exer}
\begin{proof}
If \(p,q\) are the projections on \(V,W\) respectively, then \(u=v\after p+w\after q\).
\end{proof}

\begin{exer}[11]
\(\GA(E)/E\iso\GL(E)\).
\end{exer}
\begin{proof}
Define \(\varphi:\GA(E)\to\GL(E)\) by \(\varphi(t_a\after v)=v\). Note that \(\varphi\)~is well-defined by~(3.2.17), \(\varphi\)~is a homomorphism by~(3.2.19), and \(\varphi\)~is clearly surjective. Also \(\varphi(u)=1\) if and only if \(u\)~is a translation, so \(\ker\varphi=T(E)\), the normal subgroup of translations. It follows that \(\GA(E)/T(E)\iso\GL(E)\). Finally, the mapping \(a\mapsto t_a\) is an isomorphism \(E\iso T(E)\) from the additive group~\(E\).
\end{proof}

\begin{exer}[13]
If \(u:E\to F\) is affine and \(L\)~is a variety in~\(F\), then \(\inv{u}(L)\)~is empty or a variety in~\(E\).
\end{exer}
\begin{proof}
If \(a\in\inv{u}(L)\) and \(L_0\)~is the direction of~\(L\), then \(L=u(a)+L_0\) and hence \(\inv{u}(L)=a+\inv{u}(L_0)\).
\end{proof}

\subsection*{Section~3}
\begin{exer}[3]
A necessary and sufficient condition for a nonempty subset~\(V\) of a vector space to be a variety is that for all pairs~\(x,y\) of distinct points of~\(V\), the line~\(D_{xy}\) is contained in~\(V\).
\end{exer}
\begin{proof}
The condition is necessary by~(3.3.2).

If the condition holds, choose \(v\in V\) and let \(V_0=-v+V\). We claim \(V_0\)~is a subspace, from which it follows that \(V=v+V_0\) is a variety. First, \(0=-v+v\in V_0\). If \(x\in V_0\) and \(x\ne 0\), then \(v+x\in V\) and \(v+x\ne v\), so \(D_{v,v+x}=\{\,v+\xi x\mid\xi\in\R\,\}\subseteq V\). It follows that \(\xi x\in V_0\) for all \(\xi\in\R\). If also \(y\in V_0\) and \(y\ne x\), then \(D_{v+x,v+y}\subseteq V\), so in particular \(v+2^{-1}(x+y)\in V\) and \(2^{-1}(x+y)\in V_0\). By the previous result, it then follows that \(x+y\in V_0\). Therefore \(V_0\)~is a subspace as claimed.
\end{proof}

\goodbreak
\begin{exer}[4]\
\begin{itemize}[itemsep=0pt]
\item A necessary and sufficient condition for an affine map to be a translation or a homothetic map is that its associated linear map be homothetic.
\item A necessary and sufficient condition for an affine map to preserve the direction of lines is that it be a translation or a bijective homothetic map.
\end{itemize}
\end{exer}
\begin{proof}\
\begin{itemize}[itemsep=0pt]
\item This follows from the equations \(t_a=t_a\after h_1\) and \(h_{a,\lambda}=t_{(1-\lambda)a}\after h_{\lambda}\) and \(t_a\after h_{\lambda}=h_{(1-\lambda)^{-1}a,\lambda}\) (\(\lambda\ne 1\)).
\item The condition is sufficient because such a map has the form \(t_a\after h_{\lambda}\) with \(\lambda\ne 0\), which clearly preserves the direction of lines. Conversely, suppose \(u=t_a\after v\) preserves the direction of lines. If \(x\ne 0\), let \(D\)~be the vector line through~\(x\). Then \(v(D)=D\), so \(v(x)=\lambda x\) for some \(\lambda\in\R\) with \(\lambda\ne0\), and in fact \(v(y)=\lambda y\) for all \(y\in D\). We claim \(v=h_{\lambda}\), from which the result follows. If \(y\not\in D\), then by considering the vector line~\(D'\) through~\(y\) we have \(v(y)=\mu y\) for some \(\mu\in\R\). Now \(v(D_{xy})=D_{v(x)v(y)}=D_{\lambda x,\mu y}\), and since \(v\)~preserves direction there is \(\xi\in\R\) with \(\mu y-\lambda x=\xi(y-x)\), or \((\mu-\xi)y=(\lambda-\xi)x\). Since \(y\not\in D\), this implies \(\mu=\xi=\lambda\). Therefore \(v=h_{\lambda}\) as claimed.\qedhere
\end{itemize}
\end{proof}

% References
\begin{thebibliography}{0}
\bibitem{dieudonne} Dieudonn\'e, J. \textit{Linear Algebra and Geometry.} Hermann, 1969.
\end{thebibliography}
\end{document}
